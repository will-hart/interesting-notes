\chapter{Quantitative Finance}
\label{ch:quantitative-finance}
\index{Quantitative Finance}

\newthought{This section deals with some applications of quantitative finance, 
both computational and theoretical. }

\section{Equity Mathematical Models}
\index{Equity Mathematical Models}
Equities and finance are heavily influenced by random events (see \cite{fooledbyrandomness} for a detailed philosophical discussion). As such, the analysis of financial instruments lends itself to a stochastic process.

A {\em Geometric Brownian Motion}\index{Geometric Brownian Motion} (GBM) equation (a form of stochastic differential equation, or SDE known as the Black-Sholes-Merton equation\cite{pyfinanceoreilly}) can be used to approximate a random process over time. In its continuous form this looks like \cite{advancedquantcpp}:
\begin{equation}
ds = \mu Sdt+\sigma SdW
\end{equation}
This breaks the movement of a stock price down into two key effects:

\begin{itemize}
	\item a deterministic effect (the left of the plus sign)
	\item a stochastic effect (the right of the plus sign)
\end{itemize}
In the equation, $\mu$ is known as {\it drift}, $\sigma$ is {\it volatility}, $S$ is the stock price, $dt$ is the change in time and $dW$ is an increment in a {\it Weiner process}.

As equity markets are a discrete process, this equation must be transformed into a discrete equation. 
\begin{equation}
S_{t+1}=S_t(1+r\Delta{}t+\sigma\varepsilon_t \sqrt{\Delta{}t})
\end{equation}
Here $\varepsilon$ is a sample from a gaussian distribution with zero mean and standard deviation of 1 (i.e. $N(0,1)$) and $r$ is the risk free rate of return. This equation can be solved iteratively for a given time period if $r$, $\sigma$, $\varepsilon$ and $S_0$ are provided.

Another formulation for pricing European Call options gives a similar result of the discretised SDE\cite{pyfinanceoreilly}:
\begin{equation}
S_t=S_{t-\Delta t}e^{(r-\frac{1}{2}\sigma^2)\Delta t+\sigma\sqrt{\Delta t}z_t}
\end{equation} 
In this instance, $z_t$ represents the random variable. Similar formulations can be determined for foreign exchange:
\begin{equation}
X_{t+1} = X_t(1+(r_d-r_f)\Delta{}t+\sigma\varepsilon_t\sqrt{\Delta{}t})
\label{eq:foreignexchangemodel}
\end{equation}
where $r_d$ and $r_f$ are the domestic and foreign fisk free rates of return. 



\section{Structural and Intensity Models}
\TODO See \cite{advancedquantcpp} chapter 2.

\section{Monte Carlo Simulation}
\index{Monte Carlo Simulation}
Monte Carlo simulation is a method of estimating a probabilistic outcome through a high quantity of simulations. In a quantitative finance context, we may wish to estimate the future price of an equity based on use of a GBM equation simulated $M$ times. 

For instance the process could be as follows for calculating a call option derivative based on estimate the price of a stock using Monte Carlo simulation:

\begin{enumerate}
	\item Generate $M$ different ``trajectories'' for the stock using a GBM simulation from time $t=0$ to time $t=T$. This generates a set of $N$ price estimates for $M$ different simulations, with the notation:
	\begin{equation}
		\{S_i^j\}\qquad i = 0...N,\qquad j=1...M
	\end{equation}
	This produces a vector of $M$ values for $S_T$, 
	\begin{equation}
		\{S_T^i\}\qquad i = 0...M
	\end{equation}
	
	\item We next compute the pay off for each stock value. This is given by 
	\begin{equation}
		H(S_T^i)\qquad i=1...M
	\end{equation}
	Where 
	\begin{equation}
		H(S_T)=max(S_T-K,0)
	\end{equation}
	and $K$ is the actual price of the equity or premium. The expected pay off can then be computed by the average of all pay offs:
	\begin{equation}
		E[H(S_T^i)]=\frac{1}{M}\Sigma_{i=1}^{M}H(S_T^i)
	\end{equation}
	
	\item The item should then be discounted to present value, by either applying a discount factor $DF_T$ or 
	\begin{equation}
		\pi = e^{-rT}\times E[H(S_T)]
	\end{equation}
	
	Where $\pi$ is the value of the derivative.
\end{enumerate}



\section{Binomial Trees \cite{advancedquantcpp}}
\index{Binomial Trees}

This approach builds a tree of possible prices. At each stage, the underlying can be assumed to go up or down by a given amount. The amount of change up ($u$) or down ($d$) is described by
\begin{equation}
	u=e^{\sigma\sqrt{\Delta{}t}}
\end{equation}
\begin{equation}
	d=e^{-\sigma\sqrt{\Delta{}t}}
\end{equation}
The probability of an equity going up $p$ is
\begin{equation}
p=\frac{e^{r\Delta{}t}-d}{u-d}
\end{equation}
The probability of the underlying going down is $1-p$. The binomial tree is then built in the following phases:

\begin{enumerate}
	\item Construct a tree where each level corresponds to a time step in the simulation period from $t=0$ to $t=T$. For example in two simulation steps, 
	\begin{equation}
		At = 0, S = S_0
	\end{equation}
	\begin{equation}
		At = t_1, S = uS_0,\qquad dS_0
	\end{equation}
	\begin{equation}
		At = t_2, S = u^2S_0,\qquad udS_0,\qquad udS_0,\qquad d^2S_0
	\end{equation}
	Note that the central value at the last period is shared between adjacent nodes, hence there are only three distinct estimates after $N=2$ steps.
	
	This produces a number of prices at each time step. The notation is based on the estimate number $k$ at time T. There are $N$ time steps
	\begin{equation}
		\{S_T^k\}\qquad k=1...N+1
	\end{equation}
	
	\item Once the tree has been built, the payoff $H(S_T^k)$ should be calculated for each $S_T^k$
	
	\item Finally the tree is traversed back up towards the root node, calculating the discounted weighted probability of each node. If we define a given node (not a leaf, i.e. $t\neq T$) it has two children, one for the up price (denoted $S_T^u$) and one for the down price, $S_T^d$. The value for the parent node is given by 
	\begin{equation}
	V_{T-1}^k = e^{-r\Delta{}t}[pH(S_T^u) + (1-p)H(S_T^d)]
	\end{equation}
	in this case $V_T^l$ is shorthand for $H(S_T^k)$
	
	\item Once the tree has been traversed back to the top, the value of the derivative $\pi=V_1^1$
\end{enumerate}


\section{Finite Difference Method}
\index{Finite Difference Method}

The finite difference method is a method for discretising a differential equation.\cite{advancedquantcpp} In the quantitative finance approach, we want to discretise partial differential equations (PDEs). The finite differences method is based on the relationship:
\begin{equation}
f(x) = \frac{df}{dx} \approx \frac{\Delta f}{\Delta x} = \frac{f_{i+1}-f_i}{\Delta x}
\end{equation}
The most important PDE in finance is the {\em Black-Scholes PDE}
\index{Black-Scholes PDE}, which is given by:
\begin{equation}
\frac{\delta V}{\delta t} + \frac{1}{2}\sigma^2S^2\frac{\delta^2V}{\delta S^2}+rS\frac{\delta V}{\delta S}-rV=0
\label{eq:blackscholeseq}
\end{equation}
This is usually solved in the $S$ and $t$ axes, where $S\in [a,b]$ and $t\in [0,T]$. The domain of this equation is said to be 
\begin{equation}
\Omega=\{(S,t)\forall S\in [a,b]\times t\in [0,T]\}
\end{equation}
In other words, as the finite difference method is solving some partial differential equation in $S$ and $t$, the solution space is the rectangular domain defined by the ranges of $S$ and $t$. For a European call, 
\begin{equation}
	V(S,T) = max(S-K,0)
\end{equation}
The boundary conditions are $V(a,t)=0$ and $V(b,t)=S$. This equation can be transformed with some variable substitution so that 
\begin{equation}
\frac{\delta u}{\delta\tau}=\frac{\delta^2 u}{\delta x}\qquad -\infty<x<\infty, \tau>0
\end{equation}
This is a dimensionless PDE with a new solution domain $\Omega=\{(x,\tau)\}$. The payoff relationship therefore becomes
\sidenote{Where $k=\frac{r}{0.5\times\sigma^2}$}
\begin{equation}
u(x,0) = max(e^{\frac{1}{2}(k+1)x}-e^{\frac{1}{2}(k-1)x},0)
\end{equation}
Using finite differences, the return can be described as\sidenote{Where $\alpha=\frac{\Delta\tau}{(\Delta x)^2}$}
\begin{equation}
u_{i,j+1}=\alpha u_{i+1,j}+(1-2\alpha)u_{i,j}+\alpha u_{i-1,j}
\end{equation}
This relationship can be solved iteratively, using the following steps:

\begin{enumerate}
	\item Discretise the domain into N space divisions of $dS$ and M time divisions of $dT$. Use these to determine the steps $\Delta\tau,\Delta x$.
	\item Use finite differences to approximate the derivatives
	\item Calculate the results of the equation iteratively for each time step
\end{enumerate}

For a worked example, see \cite{advancedquantcpp}, end of chapter 3.

