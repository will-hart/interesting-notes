\chapter{Market Basics}
\label{ch:marketbasics}

\newthought{This section deals with some basic stock market information and a brief introduction into fundamental stock analysis.}

\section{Description of Basic Securities}

\textbf{Bonds}\sidenote{sometimes called debt} are typically based on debt - that is bond holders lend their money to the bond issuer in return for a ``fixed income''. The fixed income is in the form of periodic interest payments on the amount loaned. At the end of the bond term the issuer repays the original amount. Bonds are considered low risk, where the issuer is a stable entity such as a government with a strong underlying economy. As a result of the low risk the return is also low. 

Bonds can be traded on the secondary market, and are usually sold at a market price which is not the same as the face value of the bond.\sidenote{Face Value is the amount to be repaid at maturity} Bonds can be ``fixed coupon'' --- which pays a fixed (percentage) amount bi-annually --- or ``capital indexed'' which pay interest quarterly based on a percentage of the face value adjusted for inflation.

\textbf{Stocks}\sidenote{also called shares or equity} allow their owners (or holders) to purchase part of a business. The business can periodically distribute some of its profit to its share holders in the form of {\it dividends}. Shareholders can also make money by the change in value of stocks as they are sold on a secondary market. Stocks have higher potential returns but also higher risks.

\textbf{Mutual funds} allow investors to pool their money to buy a combination of bonds and shares in a single fund. Depending on the mutual fund this has the advantage of built in diversification, as well as buying a portfolio of investments (hopefully) managed by an expert, with a lower cost in time and money than researching and purchasing each of the securities individually.

\section{Description of Advanced Securities}

\textbf{Derivatives} \index{Derivatives}\TODO{}

\textbf{Warrants} \index{Warrants}\TODO{}


\section{Portfolio Theory}
\index{Portfolio Theory}

A way to spread risk and achieve a certain balance of aggressive vs conservative is to maintain a portfolio of stock. For instance an aggressive investment portfolio may include 50\% in shares, 40\% in fixed income investments and 10\% in cash or equivalent. A conservative portfolio may have 70\% in fixed income, 20\% in equities and 10\% in cash and equivalents \cite{investing101}. Other sources such as Graham's {\it Intelligent Investor} suggest a 50-50 split between debt and equity, however note that this split changes with the circumstances of both the wider market and the risk aversion of the investor\cite{intelligentinvestor}.

Spreading the types and areas of investment is a strategy of {\it diversification}, which relies on various assets performing in different ways at different times. If any one class or category of security performs poorly, the idea is that other assets in a diverse portolio will maintain the value of the portfolio.


\section{Bond Pricing Analysis}
\index{Bond Pricing}

\TODO{} (\url{http://www.mysmp.com/bonds/bond-math.html} ?)

\section{Efficient Market Hypothesis}
\index{Efficient Market Hypothesis}

\TODO{}

