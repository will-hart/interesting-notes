\chapter{Photonics}
\label{ch:marketbasics}

\section{Basics}

The speed of light is given by $c=2.99\times10^8m/s$ in vacuum, and slower in other medium. Light can experience \textbf{refraction}\sidenote{deflection when passing from one medium to another} and \textbf{reflection}\sidenote{bouncing off a surface}. Detection and measurement of light energy is a field known as {\it radiometry}. The properties of light can be described by both particle and wave analogies. 

\subsection{Light as a particle}

A "light particle" is called a photon, which is a particle with no mass or charge. It carries electromagnetic energy and can interact with other particles. The amount of energy $E$ for a photon is given by 

\begin{equation}
E=\frac{hc}{\lambda}
\end{equation}
Where $h$ is Planck's constant ($6.25\times10^{-34}$), $c$ is the speed of light and $\lambda$ is the light's wavelength in meters.

The {\it photoelectric} effect gives evidence of light's particle like properties. This is the effect seen where some materials when a light is shone on them emit electrons. This behaviour reflects light as a particle because more intense radiation did not cause higher energy electrons to be emitted and electron energy was dependent on wavelength, not amplitude of the wave.

No matter how intense the light, photons below a given minimum frequency do not cause electrons to be emitted. The relationship governing the energy of emitted electrons is

\begin{equation}
E_{e-} = \frac{hc}{\lambda} - p
\end{equation}
Where $p$ is the characteristic escape energy for the given metal and $E_{e-}$ is the energy of the escaping electron.


\subsection{Light as a wave}

Light also exhibits the properties of {\it interference} and {\it diffraction} which fit the idea of a wave model of light. Waves move energy without moving mass at a speed independent of intensity or wavelength. Light waves have an electric and magnetic field which changes at right angles to the direction of motion. The wavelength ($\lambda$) is the distance between successive peaks or troughs in a wave. The wave number $\nu$ is the inverse of the wave length. Standard mathematical properties such as period $\tau$ and frequency $f$ can be calculated. 

Light waves are not always sinusoidal in shape. In addition any particle light wave may consist of a series of waves with peaks in all different directions perpendicular to the direction of travel. The amount of energy that flows across a unit area perpendicular ot the direction of travel is called the irradiance or flux density of the wave. 

Light waves may also be polarised, where the waves vibrate in specific directions perpendicular to the direction of travel. The intensity of light travelling through a linear polariser can be given by 

\begin{equation}
I(\theta) = I_0\cos^2(\theta)
\end{equation}
where $I(\theta)$ is the light intensity passed by the polariser and $I_0$ is the incident light density. 

Light waves exhibit the properties of {\it superposition}, {\it reflection}, {\it refraction}, {\it diffraction} and {\it interference}.

For refraction, the angle of incidence is the same as the angle of reflection, when measured from the surface normal. 

Refraction occurs at the interface between surfaces due to the difference in speed of light in the particular material. The ratio of speeds for two surfaces can be given by the index of refraction:

\begin{equation}
	\frac{n_2}{n_1} = \frac{\sin\theta}{\sin\phi}
\end{equation}
Where $n_2$ and $n_1$ are the indices of refraction for the two media, $\theta$ is the angle of incidence (from the surface normal) and $\phi$ is the angle of refraction (measured from the normal). 

the index of refraction can be calculated by 

\begin{equation}
n_i = \frac{c}{v_i}
\end{equation}
Where $v_i$ is the velocity of light in the medium $i$.

The slit test proved the wave like properties of light. A slit of width d has a plane wave shone on it. Where $d < \lambda$ a very even diffuse light is shone on to a surface on the far side of the slit. Essentially small spherical waves are emitted from the slit . Where $d \approx \lambda$, then emitted wave appears to have similar properties to the plane wave. 

The two slit experiment in 1801 showed the two waves of light emitted through the slits interfering, resulting in a ``zebra strip'' sort of pattern on the far surface. 

\subsection{The electromagnetic spectrum}

Only a small spectrum of electromagnetic radiation is visible light. Waves go from long wave to radio through infrared, visible light, ultraviolet, xrays and gamma rays in order of decreasing wave length. 

White light contains a mixture of different coloured light, each of which has a different wavelength. Due to their differing wavelength, the white light can be separated when refracted through a particular medium.

Any material above absolute zero emits electromagnetic radiation, with molecules having a characteristic set of spectral lines. Atoms changing state produce visible and ultraviolet radiation, whilst molecules changing vibrational or rotational states emit infrared radiation. Liquids and solids typically have much broader spectral lines than gases as they can take on a much wider range of energy states. 

At an atomic level, an atom consisting of protons and neutrons has a series of energy shells (labelled K through O) which can contain electrons. In their {\it grounded} state electrons have limited energy. However as energy is added they can become excited and move into higher energy shells. As they do this they absorb or emit quanta (unique amounts) of energy, with the exact nature depending on the electronic structure of the atom. 

For a hydrogen atom there are 6 major energy levels, ranging from $-0.38eV$ to $-13.6eV$. This means that if an electron is in the $n=3$ layer (which has an energy of $-1.5eV$ then it can emit a photon with 

\begin{equation}
	-1.51 - (-13.6) = 12.09eV
\end{equation}
If an electron is freed from the atom then its energy level $E_\infty = 0$. Atoms can also absorb photons which have energy exactly matching the difference between electron energy levels. Additionally a molecule in a gas or liquid may absorb a photon where it has a vibrational or rotational energy level matching the energy of the photon.

\subsection{Blackbody radiation}

Blackbody radiation is the theoretical maximum radiation expected for temperature related self-radiation. That is the amount of energy radiated from a body (in various spectra) based on its temperature (in Kelvin).

The energy radiated by a black body in a given wave band is the sum of all energies radiated at the wavelengths within the band. The same holds for the power emitted, which can be calculated in watts per square meter using the Stefan-Boltzmann law:

\begin{equation}
W_s = \sigma_sT^4
\end{equation}
Where $W_s$ is the radiated power, $\sigma_s$ is the Stefan-Boltzmann constant, $5.67\times10^{-8}$ and $T$ is the temperature in Kelvin. $W_s$ is the power per unit area, also known as the emitted radiant flux density. Typically graybodies do not perfectly emit and the above equation is factored down bt the emissivity ($\varepsilon$) of the material.

\begin{equation}
W_s = \varepsilon\sigma_sT^4
\end{equation}
Blackbodies emit radiation over a range of wavelengths. The power radiated per unit area $W_\lambda$ within a given waveband $\Delta\lambda$ is given by Planck's radiation formula:

\begin{equation}
W_\lambda = \frac{c_1}{\lambda^5} - \frac{1}{\frac{c_2}{e\lambda T}-1}
\end{equation}
Where 

\begin{equation}
c_1 = 2\pi c^2 h = 3.75\times10^{-16}
\end{equation}

\begin{equation}
c_2 = \frac{hc}{k} = 1.44\times10^{-4}
\end{equation}

\begin{equation}
c = 3\times10^8
\end{equation}

\begin{equation}
h = 6.626\times10^{-34}
\end{equation}

\begin{equation}
k = 1.38\times10^{-28}
\end{equation}
There is a maximum emission wavelength which can be given by Wien's displacement law:

\begin{equation}
	\lambda_{max}T = 2.898\times10^{-3}m\cdot K
\end{equation}	

\subsection{Interaction with matter}
The two main interactions are {\it absorption} and {\it scattering}. 
Absorption has been discussed as moving an electron into a higher energy level or exciting a molecule's vibration or rotation. The spectrum of absorbed light may have missing or removed wavelengths depending on what has been absorbed. This is the basis for objects having colour.

Scattering is redirection of light based on interaction with matter. Scattered radiation may have the same or longer wavelength (reduced energy) and may have a different polarisation. 

If the scatterer are significantly smaller than the wavelength $\lambda$ then they may absorb and immediately re-emit photons in a different direction. Where the emitted photon has the same wavelength as the incident light it is known as {\it Rayleigh scattering}. 

Where the wavelength of the emitted radiation is longer and the molecule is left in an excited state, it is known as {\it Raman scattering}. For Raman scattering, secondary photons may later be emitted when the molecule returns to the ground state.

Where the scatterer is of similar size or larger than the wavelength, all wavelengths of the incident light are equally scattered, a process known as {\it Mie scattering}. 

